% Unofficial University of Bridgeport Poster Template
% https://github.com/andiac/gemini-cam
% a fork of https://github.com/anishathalye/gemini

\documentclass[final]{beamer}

% ====================
% Packages
% ====================

\usepackage[T1]{fontenc}
\usepackage{lmodern}
% \usepackage[size=custom,width=120,height=72,scale=1.0]{beamerposter}
\usepackage[size=custom,width=91.44,height=60.96,scale=1.0] {beamerposter} % centimeter - Standard Landscape

%\usepackage[size=custom,width=60.96,height=91.44,scale=1.0] {beamerposter} % centimeter - UBRise Portrait

\usetheme{gemini}
\usecolortheme{UBridgeport}
\usepackage{graphicx}
\usepackage{booktabs}
\usepackage{tikz}
\usepackage{pgfplots}
\pgfplotsset{compat=1.14}
\usepackage{anyfontsize}

% ====================
% Lengths
% ====================

% If you have N columns, choose \sepwidth and \colwidth such that
% (N+1)*\sepwidth + N*\colwidth = \paperwidth
\newlength{\sepwidth}
\newlength{\colwidth}
\setlength{\sepwidth}{0.025\paperwidth}
\setlength{\colwidth}{0.3\paperwidth}

\newcommand{\separatorcolumn}{\begin{column}{\sepwidth}\end{column}}

% ====================
% Title
% ====================
\section{Header}
\title{Recycling in Seaside Park: A brief analysis on plastic polution along Long Island Sound}

\author{Huy Huong\inst{1} \and Samantha Targonski\inst{1} \and Alexandra Mones\inst{2} \and Emmanuel Ugbomah\inst{2}}

\institute[UB]{Senior Capstone} 
\section{Footer}
% ====================
% Footer (optional)
% ====================
\footercontent{
  { } \hfill
  HONR-390/CAPS-390 Senior Capstone --- Island of Misfits \hfill
  \href{mailto:john.doe@example.com}{ }}
% (can be left out to remove footer)s

% ====================
% Logo (optional)
% ====================
% Refer to https://github.com/k4rtik/uchicago-poster
% logo: https://communications.admin.ox.ac.uk/communications-resources/visual-identity/identity-guidelines/the-oxford-logo

% use this to include logos on the left and/or right side of the header:
\logoleft{\includegraphics[height=7cm]{logos/ub-logo-stacked-KO.png}}
%\logoright{\includegraphics[height=7cm]{logos/ub-logo-stacked-KO.png}}

% ====================
% Body
% ====================

\begin{document}

\begin{frame}[t]
\begin{columns}[t]
\separatorcolumn

\begin{column}{\colwidth}
\section{Abstract}
  \begin{block}{Introduction}

    Long Island Sound, a vital estuary located between Connecticut and Long Island, is an ecological and economic hub, supporting diverse marine life, recreational activities, and commercial industries. However, like many coastal regions, the "Sound" faces significant environmental challenges, one of the most pressing being the growing issue of litter. From plastic debris and fishing gear to food wrappers and cigarette butts, litter in the Sound poses threats to water quality, wildlife, and human health. The accumulation of waste not only harms marine species but also impacts local economies reliant on tourism and fishing. This research poster explores the sources, types, and consequences of litter in Long Island Sound, as well as potential solutions to mitigate this growing problem and protect this invaluable resource for future generations. \cite{STS_CT_Cleanup}

  \end{block}

  \begin{block}{Goals}

    Nam vulputate nunc felis, non condimentum lacus porta ultrices. Nullam sed
    sagittis metus. Etiam consectetur gravida urna quis suscipit.

    \begin{itemize}
      \item \textbf{Mauris tempor} risus nulla, sed ornare
      \item \textbf{Libero tincidunt} a duis congue vitae
      \item \textbf{Dui ac pretium} morbi justo neque, ullamcorper
    \end{itemize}

    Eget augue porta, bibendum venenatis tortor. 

    The FitnessGram™ Pacer Test is a multistage aerobic capacity test that progressively gets more difficult as it continues. The 20 meter pacer test will begin in 30 seconds. Line up at the start. The running speed starts slowly, but gets faster each minute after you hear this signal. [beep] A single lap should be completed each time you hear this sound. [ding] Remember to run in a straight line, and run as long as possible. 
  \end{block}

\end{column}

\separatorcolumn

\begin{column}{\colwidth}

  \begin{block}{Data Collection Methods}

    Vivamus congue volutpat elit non semper. Praesent molestie nec erat ac
    interdum. In quis suscipit erat. \textbf{Phasellus mauris felis, molestie
    ac pharetra quis}, tempus nec ante. Donec finibus ante vel purus mollis

    \begin{enumerate}
      \item \textbf{Morbi mauris purus}, egestas at vehicula et, convallis
        accumsan orci. Orci varius natoque penatibus et magnis dis parturient
        montes, nascetur ridiculus mus.
      \item \textbf{Cras vehicula blandit urna ut maximus}. Aliquam blandit nec
        massa ac sollicitudin. Curabitur cursus, metus nec imperdiet bibendum,
        velit lectus faucibus dolor, quis gravida metus mauris gravida turpis.
      \item \textbf{Vestibulum et massa diam}. Phasellus fermentum augue non
        nulla accumsan, non rhoncus lectus condimentum.
    \end{enumerate}

  \end{block}

  \begin{block}{Results}

    \begin{figure}
      \centering
      \begin{tikzpicture}
        \begin{axis}[
            scale only axis,
            no markers,
            domain=0:2*pi,
            samples=100,
            axis lines=center,
            axis line style={-},
            ticks=none]
          \addplot[red] {sin(deg(x))};
          \addplot[blue] {cos(deg(x))};
        \end{axis}
      \end{tikzpicture}
      \caption{Another figure caption.}
    \end{figure}

  \end{block}

  \begin{block}{Nam cursus consequat egestas}

    Nulla eget sem quam. Ut aliquam volutpat nisi vestibulum convallis. Nunc a
    lectus et eros facilisis hendrerit eu non urna. Interdum et malesuada fames
    ac ante \textit{ipsum primis} in faucibus. Etiam sit amet velit eget sem
    euismod tristique. Praesent enim erat, porta vel mattis sed, pharetra sed
    ipsum. Morbi commodo condimentum massa, \textit{tempus venenatis} massa
    hendrerit quis. Maecenas sed porta est. Praesent mollis interdum lectus,
    sit amet sollicitudin risus tincidunt non.

  \end{block}

\end{column}

\separatorcolumn

\begin{column}{\colwidth}

  \begin{exampleblock}{A highlighted block containing some math}

    A different kind of highlighted block.

    $$
    \int_{-\infty}^{\infty} e^{-x^2}\,dx = \sqrt{\pi}
    $$

    Interdum et malesuada fames $\{1, 4, 9, \ldots\}$ ac ante ipsum primis in
    faucibus. Cras eleifend dolor eu nulla suscipit suscipit. Sed lobortis non
    felis id vulputate.

    \heading{A heading inside a block}

    Praesent consectetur mi $x^2 + y^2$ metus, nec vestibulum justo viverra
    nec. Proin eget nulla pretium, egestas magna aliquam, mollis neque. Vivamus
    dictum $\mathbf{u}^\intercal\mathbf{v}$ sagittis odio, vel porta erat
    congue sed. Maecenas ut dolor quis arcu auctor porttitor.

    \heading{Another heading inside a block}

    Sed augue erat, scelerisque a purus ultricies, placerat porttitor neque.
    Donec $P(y \mid x)$ fermentum consectetur $\nabla_x P(y \mid x)$ sapien
    sagittis egestas. Duis eget leo euismod nunc viverra imperdiet nec id
    justo.

  \end{exampleblock}

  \begin{block}{Survery Results}

    Class aptent taciti sociosqu ad litora torquent per conubia nostra, per
    inceptos himenaeos. Phasellus libero enim, gravida sed erat sit amet,
    scelerisque congue diam. Fusce dapibus dui ut augue pulvinar iaculis.

    \begin{table}
      \centering
      \begin{tabular}{l r r c}
        \toprule
        \textbf{First column} & \textbf{Second column} & \textbf{Third column} & \textbf{Fourth} \\
        \midrule
        Foo & 13.37 & 384,394 & $\alpha$ \\
        Bar & 2.17 & 1,392 & $\beta$ \\
        Baz & 3.14 & 83,742 & $\delta$ \\
        Qux & 7.59 & 974 & $\gamma$ \\
        \bottomrule
      \end{tabular}
      \caption{A table caption.}
    \end{table}


  \end{block}

  \begin{block}{References}

    \nocite{*}
    \footnotesize{\bibliographystyle{plain}\bibliography{poster}}

  \end{block}

\end{column}

\separatorcolumn
\end{columns}
\end{frame}

\end{document}
