% Unofficial University of Bridgeport Poster Template
% https://github.com/andiac/gemini-cam
% a fork of https://github.com/anishathalye/gemini

\documentclass[final]{beamer}

% ====================
% Packages
% ====================

\usepackage[T1]{fontenc}
\usepackage{lmodern}
% \usepackage[size=custom,width=120,height=72,scale=1.0]{beamerposter}
\usepackage[size=custom,width=91.44,height=60.96,scale=1.1] {beamerposter} % centimeter - Standard Landscape

%\usepackage[size=custom,width=60.96,height=91.44,scale=1.0] {beamerposter} % centimeter - UBRise Portrait

\usetheme{gemini}
\usecolortheme{UBridgeport}
\usepackage{graphicx}
\usepackage{booktabs}
\usepackage{tikz}
\usepackage{pgfplots}
\usepgfplotslibrary{colorbrewer}
\usepgfplotslibrary{statistics}
\pgfplotsset{compat=1.18}
\usepackage{pgf-pie}
\usepackage{anyfontsize}

% ====================
% Lengths
% ====================

% If you have N columns, choose \sepwidth and \colwidth such that
% (N+1)*\sepwidth + N*\colwidth = \paperwidth
\newlength{\sepwidth}
\newlength{\colwidth}
\setlength{\sepwidth}{0.025\paperwidth}
\setlength{\colwidth}{0.3\paperwidth}

\newcommand{\separatorcolumn}{\begin{column}{\sepwidth}\end{column}}

% ====================
% Title
% ====================
\section{Header}
\title{Cleaning up Seaside Park: A brief analysis on beach pollution along Long Island Sound}

\author{Huy Huong\inst{1} \and Samantha Targonski\inst{1} \and Alexandra Mones\inst{2} \and Emmanuel Ugbomah\inst{2}}

\institute[UB]{HONR-390/CAPS-390 Senior Capstone --- Island of Misfits} 
\section{Footer}
% ====================
% Footer (optional)
% ====================

\footercontent{
  { } \hfill
  HONR-390/CAPS-390 Senior Capstone --- Island of Misfits \hfill
  \href{mailto:john.doe@example.com}{ }}
% (can be left out to remove footer)s

% ====================
% Logo (optional)
% ====================
% Refer to https://github.com/k4rtik/uchicago-poster
% logo: https://communications.admin.ox.ac.uk/communications-resources/visual-identity/identity-guidelines/the-oxford-logo

% use this to include logos on the left and/or right side of the header:
\logoleft{\includegraphics[height=7cm]{logos/ub-logo-stacked-KO.png}}
%\logoright{\includegraphics[height=7cm]{logos/ub-logo-stacked-KO.png}}

% ====================
% Body
% ====================

\begin{document}
  \begin{frame}[t]
    \begin{columns}[t]
      \separatorcolumn

      \begin{column}{\colwidth}
        \section{Abstract}
        \begin{block}{Introduction}

          Long Island Sound, a vital estuary located between Connecticut and Long Island, serves as an ecological and economic hub, supporting diverse marine life, recreational activities, and commercial industries. However, like many coastal regions, the Sound faces significant environmental challenges, with one of the most pressing issues being the growing problem of litter. From plastic debris and discarded fishing gear to food wrappers and cigarette butts, litter in the Sound poses threats to water quality, wildlife, and human health. The accumulation of waste not only harms marine species but also affects local economies that rely on tourism and fishing. While organizations like Save the Sound \cite{STS_CT_Cleanup} work across Connecticut to provide resources and organize local cleanups to reduce overall pollution, more action is needed. This research poster examines the sources, types, and consequences of litter in Long Island Sound and explores potential solutions to mitigate this issue, protecting this invaluable resource for future generations.

        \end{block}

        \begin{block}{Research Objectives}

          This research examines the volume of litter collected at Seaside Park on a single day to estimate annual waste generation and identify local sources. The study aims to determine the most common types of litter found in the park and gather survey data to develop strategies for reducing waste.

        \end{block}

        \begin{block}{Methodology}
            \begin{itemize}
              \item Surveyed the Fall 2024 Senior Capstone course on Beach Pollution
              \item Performed field research by going to Seaside Park and recording litter found along a section of the walkway.
            \end{itemize}

        \end{block}
      \end{column}

      \separatorcolumn

      \begin{column}{\colwidth}

        \begin{block}{Results}

          \begin{figure}[1]
            \centering
            \begin{tikzpicture}
                \pie[
                    radius=3.5,
                    text=legend,
                    font =\small,
                    color={cyan, orange}
                ]{
                    61.5/Summer,
                    38.5/Fall
                }
            \end{tikzpicture}
            \caption{In what season do you usually visit any beach along the Long Island Sound?}
        \end{figure}
        
        % Second Pie Chart
        \begin{figure}[2]
            \centering
            \begin{tikzpicture}
                \pie[
                    radius=3.5,
                    text=legend,
                    font =\small,
                    color={purple, 
                           pink, 
                           orange, 
                           green},
                ]{
                    7.7/Always,
                    61.5/Often,
                    23.1/Sometimes,
                    7.7/Rarely
                }
            \end{tikzpicture}
            \caption{How often do you see trash at CT beaches?}
        \end{figure}

          
          TODO: Insert Beach Cleanup data here
          \begin{table}
            \centering
            \begin{tabular}{l r r c}
              \toprule
              \textbf{Item} & \textbf{Quantity} & \textbf{Third column} & \textbf{Fourth} \\
              \midrule
              Foo & 13.37 & 384,394 & $\alpha$ \\
              Bar & 2.17 & 1,392 & $\beta$ \\
              Baz & 3.14 & 83,742 & $\delta$ \\
              Qux & 7.59 & 974 & $\gamma$ \\
              \bottomrule
            \end{tabular}
            \caption{A table caption.}
          \end{table}

        \end{block}

      \end{column}

      \separatorcolumn

      \begin{column}{\colwidth}

        \begin{block}{Discussion}

          Class aptent taciti sociosqu ad litora torquent per conubia nostra, per
          inceptos himenaeos. Phasellus libero enim, gravida sed erat sit amet,
          scelerisque congue diam. Fusce dapibus dui ut augue pulvinar iaculis.

        \end{block}

        \begin{block}{Conclusion}

          Class aptent taciti sociosqu ad litora torquent per conubia nostra, per
          inceptos himenaeos. Phasellus libero enim, gravida sed erat sit amet,
          scelerisque congue diam. Fusce dapibus dui ut augue pulvinar iaculis.

        \end{block}

        \begin{block}{References}

          \nocite{*}
          \footnotesize{\bibliographystyle{plain}\bibliography{poster}}

        \end{block}  
      \end{column}

      \separatorcolumn
    \end{columns}
  \end{frame}

\end{document}